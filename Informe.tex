\documentclass[a4paper, titlepage, 12pt]{article}
\usepackage[left=3cm, right=3cm, top=4cm, bottom=4cm]{geometry}
\usepackage[utf8]{inputenc}
\usepackage{color}
\usepackage{parskip}
\usepackage[spanish]{babel}
\usepackage{caption}
\usepackage{enumitem}
\usepackage{amssymb}
\usepackage{amsfonts}
\usepackage{amsmath}
\usepackage{tikz}
\usepackage{subcaption}
\usepackage{array}
\usepackage{hyperref}
\usepackage{graphicx}
\usepackage{float}
\usepackage[square,sort,comma,numbers]{natbib}
\usepackage{inputenc}

% Datos de la portada
\newcommand{\thesisTitle}{\LARGE Sistema de Reservación de Salas y Canchas \\
Universidad Diego Portales}
\newcommand{\thesisType}{\textbf{\textbf{\LARGE Arquitectura de Software}}}
\newcommand{\timeFrame}{Septiembre - 2025}

\newcommand{\supervisor}{ Juan Ricardo Giadach}
\newcommand{\advisor}{}

\newcommand\todo[1]{\colorbox{yellow}{#1}}
\setlength{\bibsep}{0.0pt}

\begin{document}

\begin{titlepage}
\begin{center}
\begin{figure}[h!]
    \centering
    \includegraphics[width=.65\linewidth]{udpinforlogo.png}
\end{figure}

\Huge{\textbf{\thesisTitle}}

\vspace{0.4cm}
   \rule{150mm}{0.5mm}

\vspace{0.2cm}
\normalfont{{\thesisType}\\[0.8cm]
  {\large Matias Mora Rodriguez| matias.mora1@mail.udp.cl} \\
  {\large Nicolás Contreras Silva | nicolas.contreras3@mail.udp.cl} \\
  {\large Diego Hidalgo Gallardo | diego.hidalgo1@mail.udp.cl} \\

\vspace{1cm}
\large{Fecha: Septiembre - 2025} \\

\vspace{0.8cm}
\large{\textbf{Profesor Asignado:}}\\
\supervisor\\
\newpage
\tableofcontents
\end{center}
\end{titlepage}

\newpage
\section{Equipo de Trabajo}
El proyecto será desarrollado por un equipo de tres integrantes, que trabajarán en conjunto en las etapas de análisis, diseño, construcción e implementación del sistema.
La distribución de tareas se realizará de manera colaborativa, garantizando el cumplimiento de los requisitos planteados.

\section{Descripción del Sistema y el Área}
El sistema a desarrollar corresponde a una \textbf{plataforma de reservas de salas y canchas} dentro de la Universidad Diego Portales.

Actualmente, los estudiantes y funcionarios deben realizar la reserva de canchas de forma presencial o por medio de correos, lo que genera problemas de coordinación y uso ineficiente de los espacios.
Con este sistema se busca digitalizar y centralizar el proceso de reservas, permitiendo que los usuarios gestionen sus solicitudes de manera más simple y ordenada. Al mismo tiempo, permitirá a la administración gestionar los recintos, coordinar horarios, notificar mantenciones y actualizar la disponibilidad de una manera más eficiente que la vía presencial.

\section{Objetivos del Sistema y Usuarios}
\subsection{Objetivos}
Los principales objetivos son:
\begin{itemize}
    \item Facilitar la reserva de salas y canchas mediante una plataforma en línea.
    \item Evitar la duplicidad de reservas y los conflictos de horarios.
    \item Entregar a la universidad una herramienta de gestión eficiente para sus espacios.
    \item Permitir a los usuarios consultar y administrar sus reservas en cualquier momento.
\end{itemize}

\newpage
\subsection{Usuarios}
El sistema está orientado a:
\begin{itemize}
    \item \textbf{Estudiantes:} podrán buscar disponibilidad de salas o canchas por horario, además de reservar y cancelar cuando sea necesario.
    \item \textbf{Funcionarios y profesores:} podrán reservar salas para clases, reuniones u otras actividades, de forma puntual o recurrente en caso de ser requerido, incluyendo también la posibilidad de editar o cancelar la planificación cuando corresponda.
    \item \textbf{Administradores:} tendrán la capacidad de supervisar el uso de los espacios, gestionar la disponibilidad (actualización de horarios y bloqueos) y atender incidencias, comunicando los cambios.
\end{itemize}

\section{Arquitectura del Sistema}
El sistema se implementará utilizando la \textbf{arquitectura orientada a servicios (SOA)}.
Bajo este enfoque, cada funcionalidad principal del sistema se implementará como un servicio independiente, comunicándose entre sí a través de interfaces bien definidas.

\subsection{Requerimientos Funcionales (Corregidos)}
El sistema debe cumplir con las siguientes funcionalidades básicas:

\begin{enumerate}
    \item \textbf{Autenticación de usuarios:} Validar identidad mediante credenciales institucionales (correo UDP y RUT), permitiendo inicio/cierre de sesión y renovación de token. Toda operación del sistema requiere autenticación válida y expira por inactividad.
    
    \item \textbf{Gestión de usuarios y roles (CRUD):} Crear, leer, actualizar y desactivar cuentas de estudiantes, funcionarios y administradores; los administradores pueden asignar y cambiar roles. RUT y correo son únicos; todas las altas y cambios quedan auditados.
    
    \item \textbf{Gestión de espacios (CRUD):} Permitir crear, consultar, modificar y eliminar información de salas y canchas del sistema.
    
    \item \textbf{Consulta de disponibilidad y calendario (CRUD):} Consultar la disponibilidad de salas y canchas aplicando filtros por fecha, hora y duración requerida.
    
    \item \textbf{Gestión de reservas (CRUD):} Crear, leer, modificar y cancelar reservas, gestionando estados \emph{pendiente, aprobada, rechazada, cancelada}. Valida ventana de anticipación, duración máxima, tope de reservas por usuario y \textbf{no-solape} sobre el mismo espacio de forma transaccional; la aprobación/rechazo es acción de administrador y dispara notificación y comprobante.
    
    \item \textbf{Gestión de incidencias y bloqueos (CRUD):} Registrar y consultar incidencias, actualizar su estado y aplicar bloqueos de espacio/horario. Un bloqueo impide nuevas reservas en su rango y, si afecta reservas existentes, estas se cancelan con notificación; los bloqueos impactan la disponibilidad publicada en el RF4.
    
    \item \textbf{Administración y parámetros operativos (CRUD):} Configurar y consultar parámetros globales (ventana de anticipación, tope de reservas por usuario, horarios y duración máxima). Los cambios aplican hacia futuro y quedan auditados.
    
    \item \textbf{Notificaciones:} Gestionar plantillas de notificación y enviar avisos por correo ante eventos del ciclo de reservas (creación, aprobación, rechazo, cancelación) y bloqueos por incidencia, evitando envíos duplicados e incluyendo los datos de la reserva.
    
    \item \textbf{Reportes y auditoria:} Generar reportes de uso (ocupación espacio, tasas de aprobación/rechazo, incidencias) y exponer historial de acciones relevantes (altas, cambios de estado, configuraciones), con filtros por rango temporal y exportación básica.
\end{enumerate}

\section{Modelo de Datos y Persistencia}

\subsection{Mecanismo de Persistencia Elegido}
\textbf{Decisión: Base de datos relacional (PostgreSQL)}

\textbf{Justificación de la elección:}
\begin{itemize}
    \item \textbf{Integridad referencial:} Las reservas requieren relaciones estrictas entre usuarios, espacios y horarios
    \item \textbf{Transacciones ACID:} Crítico para evitar doble reservas y mantener consistencia
    \item \textbf{Consultas complejas:} Necesarias para filtros avanzados de disponibilidad
\end{itemize}

\textbf{Alternativas descartadas y por qué:}
\begin{itemize}
    \item \textbf{NoSQL (MongoDB):} Inadecuado para relaciones complejas y transacciones críticas
    \item \textbf{SQLite:} Limitado para acceso concurrente de múltiples usuarios
\end{itemize}

\subsection{Modelo de Datos}

\textbf{Entidades principales:}

\begin{verbatim}
-- Tabla Usuarios
CREATE TABLE usuarios (
    id_usuario SERIAL PRIMARY KEY,
    rut VARCHAR(12) UNIQUE NOT NULL,
    correo_institucional VARCHAR(100) UNIQUE NOT NULL,
    nombre VARCHAR(100) NOT NULL,
    tipo_usuario ENUM('estudiante', 'funcionario', 'administrador') NOT NULL,
    activo BOOLEAN DEFAULT true,
    fecha_creacion TIMESTAMP DEFAULT CURRENT_TIMESTAMP
);

-- Tabla Espacios
CREATE TABLE espacios (
    id_espacio SERIAL PRIMARY KEY,
    nombre VARCHAR(100) NOT NULL,
    tipo ENUM('sala', 'cancha') NOT NULL,
    capacidad INTEGER NOT NULL,
    activo BOOLEAN DEFAULT true
);

-- Tabla Reservas
CREATE TABLE reservas (
    id_reserva SERIAL PRIMARY KEY,
    id_usuario INTEGER REFERENCES usuarios(id_usuario),
    id_espacio INTEGER REFERENCES espacios(id_espacio),
    fecha_inicio TIMESTAMP NOT NULL,
    fecha_fin TIMESTAMP NOT NULL,
    estado ENUM('pendiente', 'aprobada', 'rechazada', 'cancelada', 'bloqueo') 
           DEFAULT 'pendiente',
    motivo TEXT,
    fecha_solicitud TIMESTAMP DEFAULT CURRENT_TIMESTAMP,
    recurrente BOOLEAN DEFAULT false,
    patron_recurrencia VARCHAR(50),
    tipo_reserva ENUM('normal', 'bloqueo', 'incidencia') DEFAULT 'normal',
    descripcion_incidencia TEXT
);

-- Tabla Configuraciones
CREATE TABLE configuraciones (
    id_config SERIAL PRIMARY KEY,
    ventana_anticipacion_dias INTEGER DEFAULT 7,
    max_reservas_usuario INTEGER DEFAULT 1,
    duracion_max_horas INTEGER DEFAULT 4,
    hora_inicio TIME DEFAULT '08:00',
    hora_fin TIME DEFAULT '22:00'
);
\end{verbatim}

\subsection{Diccionario de Datos}

\textbf{USUARIOS:}
\begin{itemize}
    \item id\_usuario: Identificador único del usuario
    \item rut: RUT del usuario (formato 12345678-9)
    \item correo\_institucional: Email institucional UDP
    \item tipo\_usuario: Nivel de acceso (estudiante/funcionario/administrador)
\end{itemize}

\textbf{ESPACIOS:}
\begin{itemize}
    \item id\_espacio: Identificador único del espacio
    \item nombre: Nombre del espacio (Sala A, Cancha 1, etc.)
    \item tipo: Clasificación sala/cancha
    \item capacidad: Número máximo de personas que puede albergar
\end{itemize}

\textbf{RESERVAS:}
\begin{itemize}
    \item id\_reserva: Identificador único de la reserva.
    \item id\_usuario: Referencia al usuario que solicita.
    \item id\_espacio: Referencia al espacio reservado.
    \item fecha\_inicio: Fecha y hora de inicio de la reserva.
    \item fecha\_fin: Fecha y hora de término de la reserva.
    \item estado: Control de flujo de aprobación (pendiente, aprobada, rechazada, cancelada, bloqueo).
    \item motivo: Justificación ingresada por el usuario o el administrador.
    \item fecha\_solicitud: Fecha y hora cuando se realizó la solicitud.
    \item recurrente: Indica si la reserva se repite.
    \item patron\_recurrencia: Especifica frecuencia (semanal, mensual).
    \item tipo\_reserva: Tipo de reserva (normal, bloqueo, incidencia).
    \item descripcion\_incidencia: Descripción del problema cuando tipo\_reserva = 'incidencia'.
\end{itemize}

\textbf{CONFIGURACIONES:}
\begin{itemize}
    \item id\_config: Identificador único de la configuración.
    \item ventana\_anticipacion\_dias: Días mínimos de anticipación para reservar.
    \item max\_reservas\_usuario: Límite de reservas activas por usuario.
    \item duracion\_max\_horas: Duración máxima permitida por reserva.
    \item hora\_inicio: Hora de apertura operativa del sistema.
    \item hora\_fin: Hora de cierre operativa del sistema.
\end{itemize}

\section{Arquitectura SOA - Componentes}

\subsection{Estructura de Servicios}

\textbf{Servicios Backend:}
\begin{enumerate}
    \item \textbf{Servicio de Autenticación (AUTH)} - RF1
    \item \textbf{Servicio de Usuarios (USER)} - RF2
    \item \textbf{Servicio de Espacios (SPACE)} - RF3
    \item \textbf{Servicio de Disponibilidad (AVAIL)} - RF4
    \item \textbf{Servicio de Reservas (BOOK)} - RF5
    \item \textbf{Servicio de Incidencias (INCID)} - RF6
    \item \textbf{Servicio de Administración (ADMIN)} - RF7
    \item \textbf{Servicio de Notificaciones (NOTIF)} - RF8
    \item \textbf{Servicio de Reportes (REPRT)} - RF9
\end{enumerate}

\textbf{Clientes:}
\begin{enumerate}
    \item \textbf{Cliente Web Estudiantes}
    \item \textbf{Cliente Web Administradores}
\end{enumerate}

\subsection{Comunicación SOA}
Utilizando el bus de servicios proporcionado (puerto 5000) con formato:
\begin{verbatim}
NNNNNSSSSSDATOS
\end{verbatim}
Donde:
\begin{itemize}
    \item NNNNN: Longitud del mensaje (5 dígitos)
    \item SSSSS: Codigo del servicio (5 caracteres)
    \item DATOS: En JSON UTF-8
\end{itemize}

\newpage
\section{Interfaces de Componentes}

\subsection{Servicio de Autenticación (AUTH)}
\textbf{Funcionalidad:} Autenticación, renovación de token y cierre de sesión

\textbf{Interfaz:}
\begin{verbatim}
Iniciar sesión:
Entrada: 00025authrut=12345678-9&pass=secreto
Salida: 00020authtoken=abc123def456&ok

Renovar token:
Entrada: 00020authrefresh=token123
Salida: 00020authtoken=newtoken456&ok

Cerrar sesión:
Entrada: 00020authlogout=token123
Salida: 00015auth{"logout":true}

RF Satisfecho: RF1 (Autenticación de usuarios)
\end{verbatim}

\subsection{Servicio de Usuarios (USER)}
\textbf{Funcionalidad:} CRUD de usuarios, roles y auditoría

\textbf{Interfaz:}
\begin{verbatim}
Crear usuario:
Entrada: 00030usercreate={"rut":"12345678-9","correo":"user@udp.cl","tipo":"estudiante"}
Salida: 00020user{"id":123,"status":"created"}

Consultar usuarios:
Entrada: 00020usergetall
Salida: 00050user[{"id":123,"rut":"12345678-9","tipo":"estudiante"}]

Cambiar rol:
Entrada: 00025userchangerole=user=123&rol=funcionario
Salida: 00020user{"updated":true}

RF Satisfecho: RF2 (Gestión de usuarios y roles CRUD)
\end{verbatim}

\newpage
\subsection{Servicio de Espacios (SPACE)}
\textbf{Funcionalidad:} CRUD completo de salas y canchas

\textbf{Interfaz:}
\begin{verbatim}
Crear espacio:
Entrada: 00030spacecreate={"nombre":"Sala A","tipo":"sala"}
Salida: 00015space{"id":1,"status":"created"}

Consultar espacios:
Entrada: 00015spacegetall
Salida: 00050space[{"id":1,"nombre":"Sala A","tipo":"sala"}]

RF Satisfecho: RF3 (Gestión de espacios CRUD)
\end{verbatim}

\subsection{Servicio de Disponibilidad (AVAIL)}
\textbf{Funcionalidad:} Consulta de disponibilidad y calendario

\textbf{Interfaz:}
\begin{verbatim}
Consultar disponibilidad:
Entrada: 00040availfecha=2025-09-15&hora=14:00&duracion=2
Salida: 00055avail[{"id":1,"nombre":"Sala A","disponible":true,"horarios":[]}]

Consultar calendario:
Entrada: 00030availcalendar=space=1&fecha=2025-09-15
Salida: 00060avail[{"hora":"14:00","disponible":true,"reserva_id":null}]

RF Satisfecho: RF4 (Consulta de disponibilidad y calendario)
\end{verbatim}

\newpage
\subsection{Servicio de Reservas (BOOK)}
\textbf{Funcionalidad:} CRUD de reservas con validaciones y estados

\textbf{Interfaz:}
\begin{verbatim}
Crear reserva:
Entrada: 00045bookuser=123&space=1&inicio=2025-09-15T14:00&fin=2025-09-15T16:00
Salida: 00025book{"id":456,"estado":"pendiente"}

Aprobar/Rechazar:
Entrada: 00030bookapprove=reserva=456&estado=aprobada&admin=admin123
Salida: 00020book{"updated":true,"notificado":true}

Consultar mis reservas:
Entrada: 00020bookgetmyreservas=123
Salida: 00060book[{"id":456,"espacio":"Sala A","estado":"pendiente"}]

RF Satisfecho: RF5 (Gestión de reservas CRUD)
\end{verbatim}

\subsection{Servicio de Incidencias (INCID)}
\textbf{Funcionalidad:} CRUD de incidencias y bloqueos automáticos

\textbf{Interfaz:}
\begin{verbatim}
Reportar incidencia:
Entrada: 00045incidreport=space=1&tipo=mantencion&descripcion=Proyector no funciona
Salida: 00025incid{"id_incidencia":789,"estado":"abierta"}

Aplicar bloqueo:
Entrada: 00035incidblock=incidencia=789&inicio=2025-09-15T14:00&fin=2025-09-15T16:00
Salida: 00030incid{"bloqueado":true,"reservas_canceladas":3}

Resolver incidencia:
Entrada: 00030incidresolve=incidencia=789&solucion=Proyector reparado
Salida: 00025incid{"resuelta":true,"espacio_liberado":true}

RF Satisfecho: RF6 (Gestión de incidencias y bloqueos CRUD)
\end{verbatim}

\newpage
\subsection{Servicio de Administración (ADMIN)}
\textbf{Funcionalidad:} Configuración de parámetros operativos y auditoría

\textbf{Interfaz:}
\begin{verbatim}
Configurar parámetros:
Entrada: 00040adminconfig=ventana_anticipacion=7&max_reservas=1&duracion_max=4
Salida: 00020admin{"configurado":true}

Consultar parámetros:
Entrada: 00020admingetconfig
Salida: 00050admin{"ventana_anticipacion":7,"max_reservas":1,"duracion_max":4}

Consultar auditoría:
Entrada: 00030admingetaudit=fecha=2025-09-15
Salida: 00060admin[{"accion":"crear_reserva","usuario":"123","fecha":"2025-09-15T10:30"}]

RF Satisfecho: RF7 (Administración y parámetros operativos CRUD)
\end{verbatim}

\subsection{Servicio de Notificaciones (NOTIF)}
\textbf{Funcionalidad:} Gestión de plantillas y envío de notificaciones

\textbf{Interfaz:}
\begin{verbatim}
Enviar notificación:
Entrada: 00040notifsend=tipo=aprobacion&reserva=456&usuario=123
Salida: 00020notif{"enviado":true,"email":"user@udp.cl"}

Configurar plantilla:
Entrada: 00035notifplantilla=tipo=aprobacion&texto=Su reserva ha sido aprobada
Salida: 00020notif{"configurado":true}

RF Satisfecho: RF8 (Notificaciones)
\end{verbatim}

\newpage
\subsection{Servicio de Reportes (REPRT)}
\textbf{Funcionalidad:} Generación de reportes y auditoría

\textbf{Interfaz:}
\begin{verbatim}
Generar reporte de uso:
Entrada: 00030reportuso=fecha_inicio=2025-09-01&fecha_fin=2025-09-30
Salida: 00060report{"ocupacion":75,"total_reservas":150,"espacios_mas_usados":["Sala A"]}

Generar reporte de auditoría:
Entrada: 00030reportaudit=fecha=2025-09-15
Salida: 00060report[{"accion":"crear_reserva","usuario":"123","fecha":"2025-09-15T10:30"}]

RF Satisfecho: RF9 (Reportes y auditoría)
\end{verbatim}

\end{document}